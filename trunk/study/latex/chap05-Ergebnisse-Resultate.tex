%
% Masterarbeit von Marcel Weber und Silvio Wangler
% 
% 
\chapter{Ergebnisse / Resultate}
%% todo: Ergebnisse / Resultate schreiben. (Richtlinie 20 Seiten)
% Dieser Abschnitt unterscheidet sich je nach Typ der Masterarbeit
%
% Empirische sozialwissenschaftliche Arbeit
% -----------------------------------------
%
% Die Ergebnisse einer empirischen sozialwissenschaftlichen Arbeit
% k�nnen im Fall einer qualitativen oder explorativen Untersuchung
% neue Hypothesen oder Theorien sein. Im Fall einer konfirmatorischen
% Untersuchung sind die Ergebnisse getestete unter interpretierte
% Hypothesen zu stellen. Allf�llige Handlungsanweisungen, die sich aus den 
% Erkenntnissen ableiten, geh�ren ebenfalls in dieses Kapitel.

Nachfolgend wollen wir nun auf die Ergebnisse der jeweiligen Umfrage eingehen und diese gegen unsere Hypothesen stellen. Abschliessend sollen jeweils die Erkenntnisse aus den Resultaten dokumentiert und erkl�rt werden.

Einleitend kann man die R�cklaufquote der jeweiligen Umfrage aus der Tabelle \ref{tbl:Results} entnehmen.

\begin{table}[h]
  \centering
  \begin{tabular}{l|l|l}
	\hline
	\textbf{Umfrage} & \textbf{Anzahl Personen angeschrieben} & \textbf{R�cklauf} \\
	\hline
	Mitarbeiter & 250 & 56 \\	
	\hline
	Unternehmen & 50 & 17 \\
	\hline
  \end{tabular}
  \caption[R�cklauf der jeweiligen Umfragen]{R�cklauf der jeweiligen Umfragen}
  \label{tbl:Results}
\end{table}

Vorab wollen wir anmerken, dass wir sowohl bei der Mitarbeiterumfrage als auch bei der Umfrage f�r die Unternehmen erhebliche Probleme hatten eine Beteiligung von �sterreichischen Personen oder Unternehmen zu erhalten. Dies l�sst sich einerseits damit erkl�ren, dass wir praktisch keine pers�nliche Kontakte zu �sterreicherInnen haben und auch damit, dass wir �ber die Business Plattform XING\index{XING} nur bedingt �sterreichische Personen beziehungsweise Unternehmen angeschrieben haben. Daraus folgt, dass wir nachfolgend �sterreich als Land aus der Betrachtung dieser Arbeit ausschliessen, da die Resultate die in Tabelle \ref{tbl:ResultsAustria} abgebildet werden zu keinem repr�sentativen und relevanten Ergebnis f�hren w�rde. Der Vergleich beziehungsweise die Betrachtung richtet sich somit neu auf den deutschsprachigen Raum von Deutschland und der Schweiz.

\begin{table}[h]
  \centering
  \begin{tabular}{l|l|l}
	\hline
	\textbf{Umfrage} & \textbf{Anzahl Personen angeschrieben} & \textbf{R�cklauf} \\
	\hline
	Mitarbeiter & 30 & 0 \\	
	\hline
	Unternehmen & 10 & 1 \\
	\hline
  \end{tabular}
  \caption[R�cklauf aus �sterreich]{R�cklauf aus �sterreich}
  \label{tbl:ResultsAustria}
\end{table}