%
% Masterarbeit von Marcel Weber und Silvio Wangler
% 
% 
\chapter{Ergebnisse / Resultate}
%% todo: Ergebnisse / Resultate schreiben. (Richtlinie 20 Seiten)
% Dieser Abschnitt unterscheidet sich je nach Typ der Masterarbeit
%
% Empirische sozialwissenschaftliche Arbeit
% -----------------------------------------
%
% Die Ergebnisse einer empirischen sozialwissenschaftlichen Arbeit
% k�nnen im Fall einer qualitativen oder explorativen Untersuchung
% neue Hypothesen oder Theorien sein. Im Fall einer konfirmatorischen
% Untersuchung sind die Ergebnisse getestete unter interpretierte
% Hypothesen zu stellen. Allf�llige Handlungsanweisungen, die sich aus den 
% Erkenntnissen ableiten, geh�ren ebenfalls in dieses Kapitel.

Nachfolgend wollen wir nun auf die Ergebnisse der jeweiligen Umfrage eingehen und diese gegen unsere Hypothesen stellen. Abschliessend sollen jeweils die Erkenntnisse aus den Resultaten dokumentiert und erkl�rt werden.

Einleitend kann man die R�cklaufquote der jeweiligen Umfrage aus der Tabelle \ref{tbl:Results} entnehmen.

\begin{table}[h]
  \centering
  \begin{tabular}{l|l|l}
	\hline
	\textbf{Umfrage} & \textbf{Anzahl Personen angeschrieben} & \textbf{R�cklauf} \\
	\hline
	Mitarbeiter & 250 & 56 \\	
	\hline
	Unternehmen & 50 & 17 \\
	\hline
  \end{tabular}
  \caption[R�cklauf der jeweiligen Umfragen]{R�cklauf der jeweiligen Umfragen}
  \label{tbl:Results}
\end{table}

Vorab wollen wir anmerken, dass wir sowohl bei der Mitarbeiterumfrage\index{Mitarbeiterumfrage} als auch bei der Umfrage f�r die Unternehmen erhebliche Probleme hatten eine Beteiligung von �sterreichischen Personen oder Unternehmen zu erhalten. Dies l�sst sich einerseits damit erkl�ren, dass wir praktisch keine pers�nliche Kontakte zu �sterreicherInnen haben und auch damit, dass wir �ber die Business Plattform XING\index{XING} nur bedingt �sterreichische Personen beziehungsweise Unternehmen angeschrieben haben. Daraus folgt, dass wir nachfolgend �sterreich als Land aus der Betrachtung dieser Arbeit ausschliessen, da die Resultate die in Tabelle \ref{tbl:ResultsAustria} abgebildet werden zu keinem repr�sentativen und relevanten Ergebnis f�hren w�rde. Der Vergleich beziehungsweise die Betrachtung richtet sich somit neu auf den deutschsprachigen Raum von Deutschland und der Schweiz. Wir sind der Meinung dass diese Einschr�nkung der Betrachtung das Resultat der Umfrage nicht wesentlich beeinflusst und wir zu diesem Zeitpunkt immer noch zu einem relevanten Resultat kommen. Eine zweite Einschr�nkungen mussten wir bei den Wirtschaftssektoren\index{Wirtschaftssektoren} vornehmen, da der R�cklauf in landwirtschaftlichen Sektor ebenfalls gering war. Dies l�sst sich darauf zur�ckschliessen, dass wir mit einer Online Umfrage den Landwirtschaftssektor\index{Landwirtschaftssektor} nicht gerecht werden beziehungsweise diesen wahrscheinlich mit dem falschen Medium zu erreichen versuchten. Allenfalls ist das Thema f�r Landwirte, welche ihre H�fe durch die Familie\index{Familie} betreiben, ausserdem nicht unbedingt gen�gend relevant und daher konnte der erwartete R�cklauf\index{R�cklauf} nicht erzielt werden. Eine abschliessende Begr�ndung l�sst sich hierzu leider nicht finden, da wir die Landwirte im Zeitrahmen dieser Arbeit nicht �ber die Gr�nde interviewen konnten.

\begin{table}[h]
  \centering
  \begin{tabular}{l|l|l}
	\hline
	\textbf{Umfrage} & \textbf{Anzahl Personen angeschrieben} & \textbf{R�cklauf} \\
	\hline
	Mitarbeiter & 30 & 0 \\	
	\hline
	Unternehmen & 10 & 2 \\
	\hline
  \end{tabular}
  \caption[R�cklauf aus �sterreich]{R�cklauf aus �sterreich}
  \label{tbl:ResultsAustria}
\end{table}

\section{Ergebnisse aus der Unternehmensumfrage}
Im Bezug auf die Wahl zwischen der Anwendung von materiellen beziehungsweise immateriellen Anreizen hat unsere Umfrage ergeben, dass sich eine Mehrheit der befragten Unternehmen auf eine Kombination von materiellen und immateriellen Anreizen bedient. Diese Mehrheit ist sich offenbar einig, dass eine ausschliessliche Verwendung einer Anreizart nicht zwingend zu einem Ziel f�hrt. Dieses Ergebnis l�sst sich auch aus Beobachtungen aus der Praxis vergleichen. Mitarbeiter haben je nach Lebenlage unterschiedliche Bed�rfnisse, welche sie befriedigen m�chten. Bei einem angehenden Vater beziehungsweise einer angehenden Mutter k�nnen allenfalls materielle Bonuszahlungen die aktuellen Bed�rfnisse nach finanzieller Unabh�ngigkeit befriedigen. Sind die Kinder des besagten Mitarbeiters hingegen bereits im hohen Schulalter oder haben das Familienheim verlassen, k�nnten Weiterbildungsanreize oder dergleichen eher attaraktiv sein. In Abbildung \ref{fig:anreizMixUnternehmen} k�nnen die Zahlen dieser Auswertung entnommen werden.
%% todo: [siwa] Referenz auf 

\section{Ergebnisse aus der Mitarbeiterumfrage}
bla
\section{Ergebnisse im direkten Vergleich}
blinkin
\section{Erkenntnisse / Handlungsanweisungen}
wicked stuff over here.
