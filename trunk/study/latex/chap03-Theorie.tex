%
% Masterarbeit von Marcel Weber und Silvio Wangler
% 
% 
\chapter{Theorie}
%% todo: Theorie schreiben. (Richtlinie 20 Seiten)
% In diesem Abschnitt beschreiben und erkl�ren Sie theoretisch das geschilderte Ph�nomen unter Zuzug
% einschl�giger wissenschaftlicher Literatur. Beleuchten Sie die verschiedenen Aspekte des Ph�nomens,
% arbeiten Sie Widerspr�che heraus und nehmen Sie begr�ndete Stellung dazu. Dieser Abschnitt gibt die
% theoretische Antwort auf die Frage "Warum ist das Ph�nomen / das Problem so, wie es in der Einleitung
% geschildert wurde?"
%
% Tipp: F�hren Sie die Referenzen von Fakten, Evidenzen, Erkenntnissen, Theorien, Ideen usw. ab Beginn
% der Arbeit systematisch und korrekt mit.
\section{Stand der Forschung}
Die Wissenschaft der Personalentwicklung ist ein junges Gebiet, auf welchem bisher keine oder wenig empirische Forschung stattgefunden hat und daher noch in den Kinderschuhen steckt. Becker begr�ndet dies damit, dass Personalentwicklung ein Praxisthema sein. Unternehmen gestalten die Personalentwicklung entweder vom Leidensdruck festgestellter Qualifikationsdefizite getrieben als Reparaturbetrieb in den Bereichen Bildung und F�rderung. Oder sie betreiben Personalentwicklung in Entsprechung der unternehmerischen Personalpolitik als konsekutive Aktivit�t entlang der jeweiligen unternehmerischen Entscheidungen\footnote{\cite[S. 18]{becker2005}}.

\paragraph{} 
Lindert kommt in seiner Studie �ber \textit{Anreizsysteme und Unternehmenssteuerung} sogar zum Fazit, dass bez�glich Anreizsystemen die Frage des faktischen Effektivit�tsnachweises unterschiedlicher Anreizsysteme in der Forschung nur unzureichend behandelt werden\footnote{\cite[S. 14]{lindert2001}}.

\section{Unternehmensbed�rfnisse f�r Anreizsystem}
Die Personalentwicklung innerhalb des Personalmanagements hat zwei grundlegende Ziele\footnote{\cite[S.40-41]{becker2005}}. Erstens soll die Leistungsf�higkeit und die Wettbewerbsf�higkeit des Unternehmens mittels qualifizierten Mitarbeitern verbessert werden. Zweitens sollen vorhandene Potenziale der Mitarbeiter entfaltet werden.

\section{Anreizsysteme}
Diverse Anreizsysteme werden durch die verschiedenen Firmen gelebt.
\subsection{Reisen}
\paragraph{}Hauptquartier
Besuchen des Hauptquartiers, wo m�glich in anderem Land