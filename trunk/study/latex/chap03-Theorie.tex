\chapter{Theorie}
Strategisch wichtige Mitarbeiter sind ein essenzielles Standbein und ein wertvolles Gut 
einer Unternehmung. Sie verk�rpern derer Know-how und sind im t�glichen Gesch�ft nicht wegzudenken. 
Gerade in KMUs sind diese Arbeitnehmer unabdingbar und bei einer ungeplanten K�ndigung durch den 
Mitarbeiter ein grosser Verlust f�r die Unternehmung, da diese Mitarbeiter schwer zu ersetzen sind.

\begin{quote}
Replacement costs usually are 2.5 times the salary of the individual, which includes lost customers, business and damaged moral.
\footnote{Vergl. Mackay A. (2007), 1. Auflage, S. 64}
\end{quote}

Trotzdem geschieht es immer wieder, dass Mitarbeiter ihrem Arbeitgeber abspringen und zur Konkurrenz wechseln, 
sei es aus monet�ren oder anderen Gr�nden.
In der amerikanischen Hire and fire Philosphie des Personalmanagement ist es g�ngig, diesem Ph�nomen mit Geld zu entgegnen, 
d.h. den Arbeitnehmer mittels Bonus oder Lohnerh�hung zufrieden zu stellen. In den 
nachfolgenden Tabellen sieht man das Resultat einer amerikanischen Umfrage der 
Firma Chart Your Course International (siehe nachfolgende Tabellen)\footnote{Vergl. Mackay A. (2007), 1. Auflage, S. 66} 

\begin{table}[h]
\begin{tabular}{lr|lr}
\hline
Gehalt & 55\% & Angenehmer Vorgesetzter & 33\% \\
\hline
Interessante Arbeit & 53\% & Abwechslungsreiche Arbeiten & 30\% \\
\hline
Benefits & 52\% & Wertsch�tzung & 25\% \\
\hline
Angenehme Mitarbeiter & 45\% & Aus- und Weiterbildungm�glichkeiten & 19\% \\
\hline
Gutgelegener Arbeitsort & 40\% & Keine Zeit einen neuen Job zu suchen & 18\% \\
\hline
Herausfordernde Arbeit & 39\% & Karrierem�glichkeiten & 16\% \\
\hline
Flexible Arbeitszeiten & 38\% & Autonomie & 14\% \\
\hline
 & & Sozialleistungen & 5\% \\
\hline
\end{tabular}
\caption[Gr�nde weswegen Mitarbeiter in ihrem aktuellen Job bleiben]{Gr�nde weswegen Mitarbeiter in ihrem aktuellen Job bleiben}
\label{hello}
\end{table}

\pagebreak

\begin{table}[h]
\begin{tabular}{lr|lr}
\hline
Gehalt & 64\% & Schlechte Moral & 27\% \\
\hline
N�chster Karriereschritt & 57\% & Arbeitsort & 25\% \\
\hline
Hausfordernde Arbeit & 47\% & Aus- und Weiterbildungsm�glichkeiten & 25\% \\
\hline
Bessere Manager & 45\% & Unfaire Behandlung & 23\% \\
\hline
Work / Life Balance & 44\% & Finanzielle Instabilit�t der Unternehmung & 19\% \\
\hline
Identifikation mit der Mission & 40\% & Urlaub / Freizeit & 19\% \\
\hline
Bessere Benefits & 37\% & Sicherer Job & 17\% \\
\hline
\end{tabular}
\caption[Gr�nde f�r einen Jobwechsel]{Gr�nde f�r einen Jobwechsel}
\label{}
\end{table}

\paragraph{}
Diese Boni bzw. Lohnerh�hungen befriedigen den Angestellten, unserer Ansicht nach jedoch nicht nachhaltig, d.h. die erw�nschte Wirkung ist nur von kurzer Dauer ist.

\paragraph{}
Nachfolgendes Beispiel soll helfen diese These zu erkl�ren. 
Mitarbeiter A arbeitet bei der Unternehmung Z und hat ein Jahresgehalt von 100000 Franken. 
Nun erh�lt A f�r seinen Einsatz im vergangenen Jahr eine Lohnerh�hung von 10\% seines Gehalts. 
Mit diesen 10000 Franken Mehreinkommen ist A f�r die n�chsten 2 bis 3 Monate (Zeitspanne t) gl�cklich.  
Nach Ablauf von t hat sich A an das neue Gehalt gew�hnt und will mehr. Beim n�chsten oder einem sp�teren 
Lohngespr�ch nimmt A diese Erwartungshaltung ins Gespr�ch mit und entdeckt dass Unternehmung Z die 
gew�nschten Erwartungen nicht mehr erf�llen kann bzw. will. Nun muss sich A, bei gleich bleibender 
Lohnforderung, folglich entscheiden, ob er weiterhin bei Z arbeiten will oder sich eine neue Stelle 
bei der Konkurrenz von Z sucht.

\paragraph{}
Dieses Verhalten ist logisch, wenn man der Theorie von Abraham Maslow folgt. Diese besagt, 
dass Geld und Wohlstand zu den Wachstumsbed�rfnissen geh�ren, welche im Gegensatz zu den 
Defizitbed�rfnissen nie wirklich befriedigt werden k�nnen.\footnote{http://de.wikipedia.org/wiki/Maslowsche\_Bed\%C3\%BCrfnispyramide - Stand: 17.07.2007}
\paragraph{}
Nach der Zwei-Faktoren-Theorie von Frederick Herzberg gelangt ein Unternehmen, welches ihre 
Mitarbeiter stets finanziell belohnt, in die Situtation einer hohen Hygiene und geringen Motivation\footnote{http://de.wikipedia.org/wiki/Zwei-Faktoren-Theorie\_\%28Herzberg\%29\#Hygienefaktoren}.
D.h. ein Mitarbeiter ist nicht ausreichend motiviert, kann sich aber durch die finanziellen 
Zahlungen eigentlich nicht beklagen. Dies nennt Herzberg die sogenannte S�ldnermentalit�t.