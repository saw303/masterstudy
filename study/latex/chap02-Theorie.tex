\chapter{Theorie}
Strategisch wichtige Mitarbeiter sind ein essenzielles Standbein und ein wertvolles Gut 
einer Unternehmung. Sie verk�rpern derer Know-how und sind im t�glichen Gesch�ft nicht wegzudenken. 
Gerade in KMUs sind diese Arbeitnehmer unabdingbar und bei einer ungeplanten K�ndigung durch den 
Mitarbeiter ein grosser Verlust f�r die Unternehmung, da diese Mitarbeiter schwer zu ersetzen sind.

\begin{quote}
Replacement costs usually are 2.5 times the salary of the individual, which includes lost customers, business and damaged moral.
\footnote{Vergl. Mackay A. (2007), 1. Auflage, S. 64}
\end{quote}

Trotzdem geschieht es immer wieder, dass Mitarbeiter ihrem Arbeitgeber abspringen und zur Konkurrenz wechseln, 
sei es aus monet�ren oder anderen Gr�nden.
In der amerikanischen Hire and fire Philosphie des Personalmanagement ist es g�ngig, diesem Ph�nomen mit Geld zu entgegnen, 
d.h. den Arbeitnehmer mittels Bonus oder Lohnerh�hung zufrieden zu stellen. In den 
nachfolgenden Tabellen sieht man das Resultat einer amerikanischen Umfrage der 
Firma Chart Your Course International (siehe nachfolgende Tabellen)\footnote{Vergl. Mackay A. (2007), 1. Auflage, S. 66} 

\begin{table}
\begin{center}
\begin{tabular}{cccc}
Gehalt & 55\% & Angenehmer Vorgesetzter & 33\% \\
Interessante Arbeit & 53\% & Abwechslungsreiche Arbeiten & 30\% \\
Benefits & 52\% & Wertsch�tzung & 25\% \\
Angenehme Mitarbeiter & 45\% & Aus- und Weiterbildungm�glichkeiten & 19\% \\
Gutgelegener Arbeitsort & 40\% & Keine Zeit einen neuen Job zu suchen & 18\% \\
Herausfordernde Arbeit & 39\% & Karierrem�glichkeiten & 16\% \\
Flexible Arbeitszeiten & 38\% & Autonomie & 14\% \\
 & & Sozialleistungen & 5\% \\
\end{tabular}
\caption[Gr�nde weswegen Mitarbeiter in ihrem aktuellen Job bleiben]{Gr�nde weswegen Mitarbeiter in ihrem aktuellen Job bleiben}
\label{hello}
\end{center}
\end{table}