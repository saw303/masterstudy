%
% Masterarbeit von Marcel Weber und Silvio Wangler
% 
% 
\chapter{Konklusion}
%% todo: Konklusion schreiben. (Richtlinie 8 Seiten)

Nach unseren Umfragen zufolge ist ein Anreizsystem eine nicht zu weg denkende M�glichkeit, einen Mitarbeiter f�r l�ngere Zeit an ein Unternehmen erfolgreich zu binden. So viel vorne weg, es lohnt sich auf jeden Fall, sich w�hrend dem Personal Management mit dem Thema Retention auseinander zu setzen. Das best�tigt auch der R�cklauf auf unsere Umfragen. Mit 22\% beziehungsweise 34\% wird das klar best�tigt. Anf�nglich wird Aufwand voraus gesetzt der sich aber �ber die Zeit stark verringert weil etablierte Prozesse die Voraussetzung geben. Wie und wann Retention gelebt wird ist alleine dem Unternehme und dessen Philosophie �berlassen. Nachstehend wollen wir auf einige Punkte eingehen die ein Retentionmanagement f�r die Zukunft unterst�tzen soll aber keineswegs verbindlich sein sollten.

\section{Hypothese}

\begin{quote}
\textit{Immaterielle Bonussysteme binden Mitarbeiter langfristig st�rker an ein Unternehmen als wenn ausschliesslich materielle Belohnungen eingesetzt werden}
\end{quote}

Arbeitgeber sowohl Arbeitnehmer haben klar best�tigt, dass immaterielle Bonussysteme mit mehr Erfolg gekr�nt werden k�nnten, sofern sie angewendet werden. 82\% verglichen zu 59\% der Befragten geben sich klar f�r eine nicht monet�re Variante aus wobei gem�ss Arbeitgeber eine M�glichkeit der gesunden Mischung gefunden werden muss. Auf jeden Fall ist eine pers�nliche Abstimmung der Anreize die minimalste Anforderung an die Personalverantwortlichen damit sich der Mitarbeiter im Retentionprogram wohl f�hlen kann. 

\begin{quote}
\textit{Zielsetzung ist mehr als ein integrierender Teil von Anreizsystemen. Sie ist die Vorbedingung f�r die Effizienz dieser Systeme schlechthin. 
\end{quote}

Wie anfangs genannt soll eine m�glichst genau �bereinstimmung mit den kurz- und langfristigen Zielen statt finden. Unsere Mitarbeiterumfrage hat deutlich gezeigt, dass individuelle Anforderungen an den Mitarbeiter bevorzugt werden. Nach Herwig W. Kressler sollten die Ziel auf der ersten und zweiten F�hrungsstufe identisch zu den Unternehmenszielen sein. Dort wo die Verantwortung bei den funktionalen Bereichen liegt, ist es sinnvoll, spezifische Ziele zu setzen. 

Die Zielvereinbarung soll nicht zu weit gegriffen und realistisch sein. Autorit�r von "oben" auferlegt Richtlinien versprechen nicht den selben Erfolg wie wenn eine Abstimmung auch durch den Einfluss einer "bottom up" Feinabstimmung erweitert wird. 



  
% In der Diskussion (Konklusion) wird nochmals der Bogen �ber die gesamte Arbeit gespannt.
% Die Resultate werden kritisch hinterfragt und mit den Erkenntnissen aus der Literatur
% verglichen. Es wird auf L�cken hingewiesen und Empfehlungen f�r eine n�chste Untersuchung
% des behandelten Themas gegeben. Achten Sie darauf, dass Sie in diesem Abschnitt die eingangs
% gestellte Frage zusammenfassend beantworten.
Es war sch�n. Einfach nur sch�n.