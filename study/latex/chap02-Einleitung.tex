\chapter{Einleitung}
%% todo: Einleitung schreiben. (Richtlinie 8 Seiten)
% In einem ersten Abschnitt wird die konkrete Problemstellung (Ph�nomen) beschrieben. 
% Es soll klar dargelegt werden, worum es sich bei der Arbeit handelt, weshalb das Thema 
% relevant ist und welche zentrale Fragestellung in der Arbeit beantwortet wird.
%
% Tipp: Untermauern Sie die Relevanz mit Fakten und vermeiden Sie Spekulationen.
%
% Bitte beachten Sie, dass die zentrale Fragestellung, die am Ende der Einleitung stehen muss,
% durch die Arbeit auch beantwortet wird.

\paragraph{}Diese Arbeit untersucht, welche Arten immaterieller Anreizsysteme Mitarbeiter wirksam beeinflussen,
so dass ein Unternehmen langfristig auf diese Angestellten z�hlen kann. Der Fokus wird auf Incentives gelegt, 
bei welchen eine klare Bestimmung seitens Arbeitgeber liegt, d.h. es werden Systeme untersucht, welche 
nicht direkt das Sal�r des Arbeitnehmers beeinflussen. 

\section{Das Ph�nomen aus dem Wirtschaftsleben}
\section{Relevanz des Themas}
\section{Zentrale Fragestellung}