%
% Masterarbeit von Marcel Weber und Silvio Wangler
% 
% 
\chapter{Management Summary}
%% todo: Management Summary schreiben. (1-2 Seiten)
% In diesem Abschnitt wird die ganze Arbeit ausgehend vom Ph�nomen, zur Problemstellung bzw. zentralen Fragestellung �ber die verwendeten theoretischen Ans�tze, angewandten Methoden und Resultate bis zu den Schlussfolgerungen auf ein bis zwei Seiten zusammengefasst. Das erm�glicht der Lehrerschaft, sich einen raschen �berblick zu verschaffen. Das Management Summary wird am Schluss der Arbeit geschrieben und ist am Anfang der Arbeit platziert.

Gute Mitarbeiter an die Konkurrenz beziehungsweise an den offenen Arbeitsmarkt zu verlieren kostet das Unternehmen viel Geld und Zeit. Einerseits muss auf dem Arbeitsmarkt nach einem neuen Mitarbeiter gesucht und Anstellungsgespr�che gef�hrt werden. Dies kostet die verantwortlichen Personen im Management Zeit und somit dem Unternehmen Geld, da der Linienvorgesetzte in dieser Zeit nicht seine �bliche Arbeit verrichten kann. Andererseits verliert das Unternehmen Wissen, welches der ausgeschiedene Mitarbeiter mit sich nimmt und allenfalls bei der Konkurrenz gewinnbringend einsetzt.

Die Problematik <<K�ndigung durch den Mitarbeiter>> ist in jedem Unternehmen vorhanden und bekannt. Nun stellt sich die Frage mit welchen Mitteln dieser Problematik entgegengewirkt werden kann, um unn�tige finanzielle Aufw�nde t�tigen zu m�ssen und allenfalls sich gegen�ber der Konkurrenz durch Wissensverlust nicht mehr durchsetzen zu k�nnen. Genauer wollen wir mit dieser Arbeit feststellen, ob immaterielle Anreizsysteme einen Mitarbeiter nachhaltiger an ein Unternehmen bindet als dies materielle Anreize tun.

Weiter untersucht diese Arbeit, welche Arten immaterieller Anreizsysteme Mitarbeiter wirksam beeinflussen, so dass ein Unternehmen langfristig auf diese Angestellten z�hlen kann. Der Fokus wird auf Anreize (Incentives) gelegt, bei welchen eine klare Bestimmung seitens Arbeitgeber liegt. Das heisst es werden Systeme untersucht, welche nicht direkt das Sal�r des Arbeitnehmers beeinflussen.

Im Theorieteil wird im ersten Teil auf den Stand der Forschung in diesem Gebiet eingegangen. Ausserdem wird das <<Retention Management>> vorgestellt, um zu erkl�ren was der Begriff <<Retention Management>> ist und wozu es verwendet wird. Ausserdem wird auf Gr�nde der beruflichen Ver�nderungen eingegangen und es werden eine Reihe von immateriellen Anreizen mit vorgestellt.

Zusammenfassend l�sst sich aufzeigen, dass der einseitige Einsatz von Anreizen zu keiner nachhaltigen Mitarbeiterbindung f�hren kann. Immaterielle und materielle Anreize m�ssen, gut aufeinander abgestimmt, in einem unternehmensspezifischen Anreizmix zusammengesetzt und angewandt werden. Es ist ausserdem bei der Wahl des Anreizes unabdingbar auf die Bed�rfnisse des jeweiligen Mitarbeiters einzugehen und diese zu kennen. Falls eingesetzte Anreize st�rken nicht die Verbundenheit des Mitarbeiters gegen�ber dem Unternehmen, sondern bewirken eher das Gegenteil.