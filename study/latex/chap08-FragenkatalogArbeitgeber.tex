\chapter{Fragenkatalog Arbeitgeber}
Nachfolgend werden hier die Fragen an die Probandengruppe \textit{Arbeitgeber} (Personalvorgesetzte, etc. innerhalb eines Unternehmens) aufgef�hrt.

\section{Frage 1 - Firmendomizil}
In welchem Land ist der Hauptsitz ihres Unternehmens? \textit{(Nur eine Auswahl m�glich)}
\begin{itemize}
	\item Schweiz
	\item Deutschland
	\item �sterreich
	\item anderes Land
\end{itemize}

\section{Frage 2 - Firmengr�sse}
Wie viele Mitarbeiter besch�ftigt das Unternehmen? \textit{(Nur eine Auswahl m�glich)}
\begin{itemize}
	\item < 10 Mitarbeiter
	\item < 50 Mitarbeiter
	\item < 250 Mitarbeitet
	\item < 2'000 Mitarbeiter
	\item >= 2'000 Mitarbeiter
\end{itemize}

\section{Frage 3 - Wirtschaftssektor}
In welchem Sektor ist ihr Unternehmen t�tig? \textit{(Nur eine Auswahl m�glich)}
\begin{itemize}
	\item Landwirtschaftlicher Sektor
	\item Industrieller Sektor
	\item Dienstleistungssektor
\end{itemize}

\section{Frage 4 - Welche Bedeutung hat Retention Management?}
Welche Bedeutung hat Retention Management (Mitarbeiterbindung) innerhalb ihres Unternehmens? \textit{(Nur eine Auswahl m�glich)}
\begin{itemize}
	\item grosse Bedeutung
	\item mittelm�ssige Bedeutung
	\item geringe Bedeutung
	\item keine Bedeutung
\end{itemize}

\section{Frage 5 - Budget f�r Retention Management?}
Ist ein Budget f�r das Retention Management vorhanden? \textit{(Nur eine Auswahl m�glich)}
\begin{itemize}
	\item Ja
	\item Kein Budget vorhanden, jedoch ist eines vorgesehen
	\item Kein Budget vorhanden
\end{itemize}

\section{Frage 6 - Generelle oder individuelle Anwendung}
Verwenden Sie generelle oder individuelle Anreize zur Mitarbeiterbindung? \textit{(Nur eine Auswahl m�glich)}
\begin{itemize}
	\item Generell, f�r alle die gleichen Anreize
	\item Anreize werden individuell pro Mitarbeiter angewandt
	\item Beides
\end{itemize}

\section{Frage 7 - Seit wann wird Retention Management eingesetzt?}
Seit wann wird in Ihrem Unternehmen Retention Management eingesetzt? \textit{(Angabe der Anzahl Monate)}
\begin{itemize}
	\item Anzahl Monate
\end{itemize}

\section{Frage 8 - Wirksamkeit von Retention Management}
F�r wie wirksam erachten Sie den Einsatz von Retention Management innerhalb Ihres Unternehmens? \textit{(Nur eine Auswahl m�glich)}
\begin{itemize}
	\item Sehr wirksam
	\item m�ssig wirksam
	\item nicht wirksam
\end{itemize}

% Allgemeine Fragen zu Anreizen

\section{Frage 8 - Materielle oder immaterielle Anreize?}
Welche Anreize setzen Sie zur Mitarbeiterbindung ein? \textit{(Nur eine Auswahl m�glich)}
\begin{itemize}
	\item Materielle Anreize (Lohnerh�hnung, Boni, etc.)
	\item Immaterielle Anreize (Aus- \& Weiterbildung, Coaching, Besuch von Konferenzen, etc.)
	\item Beides. Wir setzen sowohl materielle als auch immaterielle Anreize ein.
\end{itemize}

\section{Frage 10 - Zyklus}
Innerhalb von welchen Zyklen machen Sie sich �ber die Mitabeiterbindung im Unternehmen Gedanken bzw. sitzen mit Mitarbeitern bzgl. Mitarbeiterbindung zusammen?
\begin{itemize}
	\item w�chentlich
	\item monatlich
	\item quartalsweise
	\item halb j�hrlich
	\item j�hrlich
	\item > 1 Jahr
\end{itemize}

% materieller Pfad
\section{Frage 11 - Weswegen werden nur materielle Anreize eingesetzt?}
Weswegen werden nur materielle Anreize eingesetzt? \textit{(Mehrere Antworten m�glich)}
\begin{itemize}
	\item Gute Erfahrungen mit materiellen Anreizen
	\item Kosten sind geringer
	\item Keine Erfahrungen mit immateriellen Anreizen
	\item Schlechte Erfahrungen mit immateriellen Anreizen
	\item Widerspricht der Unternehmenspolitik
\end{itemize}

\section{Frage 12 - Welche materiellen Anreize finden Anwendung?}
Welche materiellen Anreize werden in der Unternehmung angewandt?

\begin{itemize}
	\item (1) Lohnerh�hung
	\item (2) Lohn mit variablem, ergebnisbezogenem Anteil
	\item (3) Bonus
	\item Kombinationen von 1, 2 und 3.
	\item andere
\end{itemize}

\section{Frage 13 - Materielle Favoriten}
Welche der drei Anreize w�rden Sie als Ihren Favoriten bezeichnen?

\begin{itemize}
	\item Lohnerh�hung
	\item Lohn mit variablem, ergebnisbezogenem Anteil
	\item Bonus
\end{itemize}

% immaterieller Pfad
\section{Frage 14 - }
\begin{itemize}
	\item
\end{itemize}


\section{Questionflow}